% Dokumenttyp
\documentclass[a4paper,11pt,twoside,german]{article}

% Befehle für Seitenränder
\newcommand{\largeborder}{3cm}
\newcommand{\smallborder}{2cm}
\newcommand{\topborder}{1cm}
\newcommand{\bottomborder}{1cm}

\usepackage[
    left=\largeborder,
    right=\largeborder,
    top=\topborder,
    bottom=\bottomborder,
    includeheadfoot,
    ]{geometry}
    
% \textwidth 16cm
% \textheight 24cm
%  \topmargin -15mm
% \setlength{\oddsidemargin}{-5mm}
% \setlength{\evensidemargin}{5mm}


%%%%%%%%%%%%%%%%%%%%%%%%%%
%%% Benötigte Packages %%%
%%%%%%%%%%%%%%%%%%%%%%%%%%

%%% Sprache %%%
\usepackage[ngerman]{babel}    	% damit man deutsche Dinge tun kann
\usepackage{ziffer} % Deutsche Zahlen (kein Abstand hinterm Komma, Punkt als Tausendertrennzeichen, etc...)
\usepackage[utf8]{inputenc}   % damit Umlaute funktionieren
\usepackage[autostyle=true,german=quotes]{csquotes} % Anführungszeichen richtig machen
\usepackage{blindtext} % damit man sinnlosen Text produzieren kann

%%% Grafiken %%%
\usepackage{graphicx}    % damit man Bilder einbinden kann
\usepackage{float}    	% damit Floats da sind, wo sie hingehören
\usepackage[font=footnotesize,labelfont=bf,skip=0pt]{caption} % Captions anpassen
\addtolength{\intextsep}{5mm} % mehr Platz um figures und tables
\usepackage{subcaption}


\usepackage{amsmath}    	% damit abgefahrene mathematische Dinge in equations gehen
\usepackage{enumitem}   % Damit Aufzählungen hübscher sind
\usepackage{ifthen}     % Damit man if-else-Strukturen machen kann

\setlength{\parindent}{0pt} % Keine Absatzeinrückung
\usepackage{setspace} % Zeilenabstände machen können
\linespread{1.1} % Zeilenabstand definieren
\newcommand{\absatz}{\smallbreak} % Standardmäßig \lb zur Trennung von Absätzen verwendet und hier das Kommando anpassen
%\renewcommand{\familydefault}{\sfdefault} % Serifenlose Schrift

%%% Links und Verweise %%%
\usepackage[breaklinks]{hyperref} % Damit Links funktionieren
\hypersetup{colorlinks=false,pdfborder={0 0 0}} % Damit die hässlichen Kästen um Links weg sind


%%% Bibliographie %%%
\usepackage[authoryear,round]{natbib} % Der Stil der LiteraturVERWEISE im Text
\bibliographystyle{dcu-german-spec} % Der Stil des LiteraturVERZEICHNISSES
%\bibliographystyle{dcu} % Der Stil des LiteraturVERZEICHNISSES


% Makro für Bibliographie
\newcommand{\literaturverzeichnis}[1]{
    \renewcommand{\harvardand}{und} % ein UND statt AND bei mehreren Autoren
    \bibliography{#1}
    }

\usepackage[nottoc,notlot,notlof]{tocbibind} % damit das Literaturverzeichnis im Inhaltsverzeichnis auftaucht, aber NICHT nummeriert wird
% \usepackage[nottoc,numbib]{tocbibind} % damit das Literaturverzeichnis nummeriert wird und im Inhaltsverzeichnis auftaucht


% Commands to place things on a left or right side
% Put this before something you want to place either right (onto the NEXT EVEN page) or left (onto the NEXT ODD page)
\newcommand{\emptypage}{\newpage\leavevmode\thispagestyle{empty}\newpage}
\newcommand{\fillwithemptypagestillevenside}{\ifthenelse{\isodd{\thepage}}{\emptypage}{}}
\newcommand{\fillwithemptypagestilloddside}{\ifthenelse{\isodd{\thepage}}{}{\emptypage}}

\newcommand{\breaktoevenside}{\ifthenelse{\isodd{\thepage}}{\newpage}{}}
\newcommand{\breaktooddside}{\ifthenelse{\isodd{\thepage}}{}{\newpage}}

% Metadaten
\usepackage{titling}
\title{LEX extended abstract}
\author{
    Büchau, Yann Georg \\
    \small{\texttt{64\,36\,211}}
    \and
    Finn, Tobias Sebastian \\
    \small{\texttt{00\,00\,000}}
    \and 
    Schaper, Maximilian \\
    \small{\texttt{00\,00\,000}}
    }
%%%%%%%%%%%%%%%%%%%%%%%%%%%%%%%%%%%%%%%%%%
\begin{document}
\raggedbottom


%%%%%%%%%%%%%%%%%%
%%% Titelseite %%%
%%%%%%%%%%%%%%%%%%

\makeatletter
\begin{titlepage}

%\newgeometry{
%    left=\smallborder,
%    right=\smallborder,
%    top=\topborder,
%    bottom=\bottomborder,
%    includeheadfoot,
%    }


\vspace*{\fill}
\begin{center}
\Large{\textbf{Lehrexkursion Fehmarn}}\\
\large{29.08.2016 - 09.09.2016}\\
\vspace{5mm}
\Large{\textbf{Arbeitsgruppe\\Wolkenkamera Stereo}}\\
\vspace{1cm}

% authors
\begin{large}
\begin{tabular}{ccc}
Büchau, Yann Georg & Finn, Tobias Sebastian & Schaper, Maximilian \\
\small{\texttt{64\,36\,211}} & \small{\texttt{00\,00\,000}} & \small{\texttt{00\,00\,000}}
\end{tabular}
\end{large}

\vspace{1cm}
\large{Meteorologisches Institut}\\
\large{Universität Hamburg}\\
\vspace{1cm}
\large{{\today}}
\end{center}
\vspace*{\fill}

\clearpage
% \restoregeometry


\end{titlepage}
\makeatother

\newpage


%%%%%%%%%%%%%%%%%%%
%%% Kurzfassung %%%
%%%%%%%%%%%%%%%%%%%
\thispagestyle{empty} % Don't show page number this page

\section*{Kurzfassung}

Das ist der Abstract.
\blindtext[3]

\newpage


%%%%%%%%%%%%%%%%%%%%%%%%%%
%%% Inhaltsverzeichnis %%%
%%%%%%%%%%%%%%%%%%%%%%%%%%
\setcounter{page}{1} % Page counting begins here!

\tableofcontents % Inhaltsverzeichnis
\vspace*{\fill}

%%%%%%%%%%%%%%%%%%%%%%%%%%%%%
%%% Abbildungsverzeichnis %%%
%%%%%%%%%%%%%%%%%%%%%%%%%%%%%
\listoftables % Abbildungsverzeichnis

\listoffigures % Abbildungsverzeichnis

\newpage


%%%%%%%%%%%%%%%%%%
%%% Motivation %%%
%%%%%%%%%%%%%%%%%%
\section{Motivation}

Ceilometer sind teuer, Kameras sind billig und vielseitig (Bedeckungsgrad, Höhenwind, Wettergeschehen, etc...).
Es lohnt sich daher die Untersuchung, inwiefern die Wolkenhöhe auch mit Kameras bestimmt werden kann.
\absatz
\blindtext[1]

%\newpage

%%%%%%%%%%%%%%
%%% Geräte %%%
%%%%%%%%%%%%%%
\section{Geräte}

Die verwendeten zwei Kameramodelle haben beide die Bezeichnung VIVOTEK FE8174V.
Metadaten sind in der Tabelle \ref{TabelleKameraMeta} zu finden.
\absatz

\begin{table}[!h]
\begin{center}
\caption{Metadaten der verwendeten Kameras}
\label{TabelleKameraMeta}
\absatz
\begin{tabular}{|c|c|c|}
\hline
\textbf{Kamera} & \textbf{GPS Position} & \textbf{Höhe über N.N.[m]} \\\hline
\textbf{Zaun}   & $54,4947^\circ\,\mathrm{N}$ \, $11,2408^\circ\,\mathrm{O}$ & 9                \\\hline
\textbf{Acker}   & $54,4959^\circ\,\mathrm{N}$ \, $11,2377^\circ\,\mathrm{O}$ & 0                \\\hline
\end{tabular}
\vspace{-0.5cm}
\end{center}
\end{table}

\blindtext

%\newpage % Neue Seite 

%%%%%%%%%%%%%%%%%%%
%%% Kalibration %%%
%%%%%%%%%%%%%%%%%%%
\section{Kalibration}

Grundlegend für Triangulation auf Basis von Kamerabildern ist die Kenntnis des Azimuth- und Elevationswinkels in jedem Bildpunkt. 
Dazu ist eine Kalibration der verwendeten Kameras erforderlich, mit der sowohl die radiale Projektion der Kameralinse als auch die Drehungseffekte durch ungenaues Aufstellen der Kamera quantifiziert werden können.
Für eine derartige Kalibration sind Raumpunkte nötig, deren Positionen sowohl auf dem Bild als auch in der Realität bekannt sind.
Es bietet sich hierfür die Sonne an, deren Azimuth- und Elevationswinkel zu jedem Zeitpunkt berechenbar sind.
Legt man an einem nahezu wolkenlosen Tag eine Goldfolie auf die Kamera und passt einige Kameraeinstellungen an, erscheint lediglich die Sonne als heller Fleck auf dem ansonsten dunklen Bild.
Das ermöglicht die programmatische Bestimmung der Sonnenposition auf dem Bild.


%\newpage

%%%%%%%%%%%%%%%%%%%%%%%
%%% Wolkenerkennung %%%
%%%%%%%%%%%%%%%%%%%%%%%
\section{Wolkenerkennung}

Wolkenerkennung.
Machste halt k-Means...
\blindtext

%%%%%%%%%%%%%%%%%%%%%%%%
%%% Wolkenklustering %%%
%%%%%%%%%%%%%%%%%%%%%%%%
\subsection{Wolkenklusterung}

\blindtext[3]



%%%%%%%%%%%%%%%%%%%%%%
%%% Wolkenmatching %%%
%%%%%%%%%%%%%%%%%%%%%%
\subsection{Wolkenwiedererkennung}

Wolkenwiedererkennung.
\blindtext[3]


%\newpage


%%%%%%%%%%%%%%%%%%%%%%%
%%% Höhenbestimmung %%%
%%%%%%%%%%%%%%%%%%%%%%%
\section{Wolkenpositionierung}

Doppelanschnitt.
\blindtext[2]

%\newpage

%%%%%%%%%%%%%%%%%%%
%%% Validierung %%%
%%%%%%%%%%%%%%%%%%%
\section{Validierung}

Drachen und Ceilometer.
\blindtext[3]


%\newpage


%%%%%%%%%%%%%
%%% Fazit %%%
%%%%%%%%%%%%%
\section{Fazit}

Zusammenfassung.
\blindtext[3]

%\newpage % Neue Seite

%%%%%%%%%%%%%%%%%%%%%%%%%%%%
%%% Literaturverzeichnis %%%
%%%%%%%%%%%%%%%%%%%%%%%%%%%%
\literaturverzeichnis{bibliography.bib} % 



\end{document}
